\documentclass[dvipdfmx,autodetect-engine]{jsarticle}
\usepackage{tikz}

\title{課題7}
\author{斎藤 宏太}

\begin{document}
\maketitle
\section{kadai7-2.c}
ニュートン法を用いた式で考えると、
f(x)=0の解aに近い値x1をきめ、x2以降は次のようになる\\
x(n+1)=x(n)-(f(x(n))/df(x(n)))とnを大きくすることで初項が何であっても、近似値aが求まる\\
\end{document}